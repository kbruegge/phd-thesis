\begin{center}
  \large{Abstract}
\end{center}
%
\noindent

% This text deals with the data analysis process of imaging atmospheric Cherenkov telescopes (IACT) with a focus on reproducibility and open-source solutions.
Imaging atmospheric Cherenkov telescopes (IACT) observe the sky in the highest energy ranges. 
% These telescopes capture gamma rays originating in the most extreme environments in the universe. 
From the remnants of cataclysmic supernovae to jets powered by supermassive blackholes in the center of distant galaxies,
IACTs can capture the light emerging from the most extreme sources in the universe. 

% The first part of my thesis mainly deals with the creation of energy spectra of point-sources as observed by IACTs.
With the recent advent of multi-messenger astronomy it has become critical for IACTs to publicly share their data and software.
% The first part of my thesis motivates the need for a common description of instrument response functions and event lists 
For the first time since the inception of IACT technology, in a combined effort of the \hess, \magic, \veritas, and \fact collaborations,
observations of the Crab Nebula were made available to the general public in a common data format. 
The first part of my thesis demonstrates the viability of the common data format 
by performing a spectral analysis of the Crab Nebula on the published datasets.
% The majority of the text deals with the computation of energy spectra and flux-point estimates from this common dataset. 
% of the Crab Nebula and 
% The text aims to work as an introduction for the uninitiated reader and tries to collect and formalize
% 
The text gives detailed descriptions and mathematical formalizations of instrument response functions (IRFs) and the statistical 
modeling used for typical spectral analyses. This is essential to understand the measurement process of IACTs. 
The ultimate goal of this part of the thesis will be to use Hamilton Markov Monte Carlo methods 
to test spectral models and unfold flux-point estimates for the Crab Nebula.

The common data format paves the road for the operation of the upcoming Cherenkov Telescope Array (CTA).
Once CTA has been constructed, it will be the largest and most sophisticated experiment in the field of ground-based gamma-ray astronomy.
It will be operated as an open observatory allowing anyone to access the recorded data.
The second part of my thesis concentrates on reproducible analysis for the Cherenkov Telescope Array (CTA).
Once operational, CTA will produce a substantial amount of data creating new challenges for data storage and analysis technologies.
In this part of the thesis I use simulated CTA data to build a comprehensive analysis chain based on fully open-source methods.
The goal is to create a pipeline that rivals the physics performance of CTA's closed-source reference implementation.
Every step of the analysis, from raw-data processing to the calculation of sensitivity curves,
will be optimized with respect to complexity, reproducibility and run-time.

% % \section{Thesis Contents and Current Progress}

% My thesis deals with the data analysis process of imaging atmospheric Cherenkov telescopes (IACT) under resource constraints
% with a focus on reproducibility.
% For my thesis I use simulated data of the CTA experiment to build an exhaustive analysis chain based on open-source software and
% reproducible methods. Every step of the analysis, from raw-data processing to estimation of spectral energy densities,
% will be optimized with respect to real-time constraints, complexity, and reproducibility.
% All methods and algorithms will be tested using open data from the currently operating Cherenkov telescopes


% The astroparticle community inhereted  many of its ideas and technologies from 
% In the first part of this thesis I lay out a way to perform high-level spectral analysis of IACT data using open-source programs and data formats.
% The Crab Nebula is the prototypical object of this kind. It is a steady source of bright TeV gamma-ray emission in the northern sky and is
% the model to data from six different experiments using Markov chain sampling. This empha- sizes the importance of open data in the Cherenkov astronomy community.
%  Without open access to flux data from multiple wavelengths, no model assumptions can be validated.
% Chapter 5 goes through the all the harrowing details needed in order to understand the measurement process of IACTs. 
% A critical part of which is the computation of the instrument response functions. In section 5.2 these response functions are formalized in an instrument agnostic way. 
% The ultimate goal of which, is the construction of a common data format that can be used by any IACT instrument.
%  Traditionally, all IACT collaborations use their own proprietary software and data formats to produce high-level results like flux points.
  % In a mutual endeavour between the MAGIC, H.E.S.S, VERITAS, and FACT collaborations to change the current state of affairs, a small data sample of Crab Nebula
  %  observations was made public in this common data format. Using this data we published the first joint and fully open-source analysis of combined IACT data [121]. 
  %  In section 5.6 I use the published event lists and instrument response functions to fit a log-parabolic spectral model to the IACT data using Hamilton Markov chain
    % sampling. Section 5.7 shows how that sa
% \todo{diesen Mist vernünftig schreiben}

% \begin{center}
%   \textsc{Kurzfassung}
% \end{center}
% %
% \noindent
% %

% \begin{german}
% Meine Thesis is hard am dealen.
% \end{german}
