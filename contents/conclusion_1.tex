\chapter{Conclusion}

Supernova remnants like the Crab Nebula are the driving forces of cosmic-ray acceleration in our own galaxy.
These objects emit electromagnetic radiation in a vast range of wavelengths. Observation from radio, infrared and optical 
to X-ray and very-high-energy gamma ray are combined into the objects' characteristic spectral energy distribution. 
Most of the observed light is due to synchrotron emission as discussed in \cref{sec:synchro}.
Cherenkov telescopes, however, observe the inverse Compton emission of these sources. 
% A log-parabolic spectral energy distribution of electrons leads to the characteristic 
A log-parabolic energy distribution emerges from simple stochastic arguments as shown in \cref{sec:log-par-he}. 
This is compatible with the spectral shape of the inverse Compton emission in SNR
as measured by IACTs shown in \cref{fig:sed_fit_he}.
In order to build a spectral model for the Crab Nebula I assumed a single electron spectrum and homogenous magnetic field strength.
I modeled the synchrotron, inverse Compton, and \ssclong emission of the Crab Nebula 
using the open-source \naima software.
\Cref{fig:ssc_fit} shows that this simple model fits the X-ray and gamma-ray data well. The resulting magnetic field in 
the Crab Nebula was estimated to be $B = \input{build/naima_results/param_5_raw.txt} \quad \si{\micro\gauss}$. 

Open software and common data formats are the key ingredients to reproducible science.
In a joint undertaking between the \fact, \magic, \veritas, and \hess telescope collaborations, Crab Nebula data was made public in 
a common data format. This open dataset contains the event lists and instrument response functions for observations of the Crab Nebula. 
In \cref{sec:spectral_fit} I fitted a log-parabolic spectral model to the observations in this open dataset. 
The statistical model that describes the measurement process
assumes Poissonian distributions for the events in both the signal and background region of the sky. 
The expected values for the distribution have to be calculated from the telescope's instrument response function.
The full model, taking into account all nuisance parameters, was given in \cref{eq:full_ll}. 
The posterior probability of the model is then sampled using Hamilton Markov chain techniques. 
In order to use Hamiltonian sampling, the gradient of the posterior with respect to the free parameters 
needs to be calculated. 
This includes the integral term over the energy spectrum and the application of the instrument response.
The analytical solution to the integral of the log-parabolic spectrum as given in \cref{eq:integral_solution}
suffers from numerical problems.
It is possible to circumvent these numerical difficulties by approximating the integral with the trapezoidal 
rule. I then use \theano's automatic differentiation techniques to find the gradient for the posterior. 
Hamiltonian sampling is performed by the \pymc, which allows me to sample the model several hundred times per second and 
CPU core.
The resulting spectra are compared to the full SSC model in \cref{fig:pymc_fit_spectrum}. 
\Cref{sec:unfolding} demonstrates how the same sampling techniques and the same model assumptions can be used to 
to unfold the flux points for each individual telescope. 

In the future the \pymc project will change its backend from the now defunct \theano library to \emph{TensorFlow Probability}~\cite{tensorflow-prob}.
This might allow for even quicker sampling, which becomes important when thousands of distinct observations are being considered.
Once the Cherenkov Telescope Array is deployed, this will become a relevant use case.

CTA will be the first experiment in 
high-energy physics that operates as a public observatory. The large user community requires an open and programming-language agnostic 
format definition. 
While the small data sample is a great test bench for new software and algorithms, 
more observational data must be made public in the future.

