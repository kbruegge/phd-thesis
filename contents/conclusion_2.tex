\chapter{Conclusion}

The deployment of the Cherenkov Telescope Array project will ring in a new era of Cherenkov astronomy. 
Its unprecedented size brings new challenges to every aspect of traditional IACT analysis. 
This second part of my thesis shows that a fully open, reproducible and configurable 
analysis pipeline matches the performance of the previous CTA reference analysis. 

The development of \ctapipe as an open-source tool is a paradigm shift in the history of very-high-energy physics. 
The \ctapipe project is a chance to bundle the expert knowledge from multiple telescope collaborations
into a single place. I contributed to \ctapipe implementation that performs the directional reconstruction 
of air showers. The resulting angular resolution as seen in \cref{fig:ang_res_optimized}
closely follows that of the reference analysis. 
To ensure reproducibility for the processing, I used the methods provided by \ctapipe 
to implement a configurable pipeline which reads simulated CTA data and performs all steps necessary
to perform background suppression and energy estimation. 
As of yet, no official data format has been agreed on to store the results produced by \ctapipe. 
I used a column-based storage with unique identifiers on each row that allows 
me to perform database-like queries on the data.
We developed the \aicttools package to handle the common machine-learning tasks encountered in IACT analysis.
Efficient background suppression and energy estimation is maybe the most challenging part of any IACT analysis.
In order to supply the multivariate methods with as much information as possible, the models were trained 
using per-telescope as well as array-wide features. 
The \aicttools support this by joining two tables before handing them to the model.
For application to CTA data special care needs to be taken in order to ensure data consistency
as the data can only be read in a batch-wise manner.
The \aicttools perform remarkably well on CTA data as shown in \cref{fig:roc}. 
% As argued in the text, it is not easy to compare results between different analyses at this point. 
% Especially considering the fact that I use no pre-selection criteria to remove 

The final benchmark for any CTA analysis is the sensitivity curve.
The reference values published by CTA are computed with proprietary software from the \magic and \veritas 
collaborations, for which no detailed documentation is available. 
I built an open-source toolkit to compute sensitivity curves and detection significances on a per-event basis.    
The resulting sensitivity curve is shown in \cref{fig:sensitivity}. 
The image shows that the analysis I developed for my thesis performs just as well or event better than the reference. 

The future development of \ctapipe needs to concentrate on the definition of file formats so that 
an entire processing pipeline can be implemented. Once that is achieved, an official benchmark has to be defined 
in order to make different analyses comparable with each other. 
The performance of the \aicttools on CTA data could be improved in a future iteration by applying two nested models on the data: one on 
a per-telescope level, and one that summarizes the result of the single-telescope predictions for the entire array-event. 
A new production of CTA simulations has recently been started. This simulation has been adapted with data from the prototype 
telescopes that have been deployed during the last two years. 
CTA now has the chance to compute the next official performance numbers using a fully reproducible and open-source pipeline. 