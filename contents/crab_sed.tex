\chapter{Modeling the Crab Nebula Emission}
\label{ch:crab-sed}
Modeling the broadband photon emission of gamma-ray sources is an important tool to study the driving forces behind cosmic-ray acceleration.
Accurate modeling can give valuable clues about the magnetic fields present in these sources as well as matter distribution and 
composition. 
Having access to multi-wavelength data from many instruments is pivotal to SED modeling. Unfortunately collecting this data is 
often tedious if not outright impossible. These data points are often proprietary and not available online. Even data published 
in journal papers is often solely provided as a plot without accompanying tables or data files. This problem will become 
an even larger problem in the future of multi-messenger astronomy. In an effort to change the status quo, flux data of the Crab Nebula has
been collected by \cite{meyer_2010,hegra-crab-data,buehler-crab-data} and published within the open-source \gammapy project.
I used this data to evaluate a simple model of the acceleration processes in the Crab Nebula. 
Previous analyses~\cite{meyer_2010,atoyan_crab} indicate that two distinct electron populations are present in the source. 
In this case I am only interested in the high-energy part of the photon spectrum above \SI{E7}{\eV}. Here, a single 
electron population suffices to describe the data. I use a \ssclong model to calculate
the gamma-ray flux given the shape of the electron distribution and the magnetic field strength. 
I assume that the magnetic field is of constant strength $B$ and isotropic throughout the entire acceleration region.
The electron population is modeled to be distributed according to a log-parabolic energy spectrum 
\begin{equation}
  \label{eq:electron_model}
    N(E) =H(E,\, E_{\text{min}}, \, E_{\text{max}}) A \left( \frac{E}{E_0} \right)^{-\alpha -\beta \log_{10}\left(\frac{E}{E_0}\right)},
\end{equation}
where $H$ is a step function describing a cutoff beyond maximum and minimum electron energies of $E_\text{min}$ and $E_\text{max}$.
\begin{equation*}
  \label{eq:step_function}
    H(E) = {
      \begin{cases}
        E : & E < E_{\text{min}} \; \text{and} \; E \geq E_{\text{max}} \\
        0 : & \text{otherwise} \\
      \end{cases}
    }
\end{equation*}
The electrons produce synchrotron emission in the nebula's magnetic field as described in~\ref{sec:synchro}
and inverse Compton emission on existing photon fields as described in~\ref{sec:ic}.
Following~\cite{atoyan_crab}, four photon fields are assumed to seed the IC process. First, there is the photon field of
the cosmic microwave background (CMB), which is modeled as blackbody radiation with a temperature of \SI{2.7}{\kelvin}.
Observations in the optical and infrared regime show glowing filaments of gas and dust inside the nebula. 
This second photon field is also assumed to be an isotropic blackbody radiator. The dust has an approximate temperature of \SI{70}{\kelvin}
and an energy density of \SI{0.5}{\eV \per \cubic \centi \metre}.
The third field is due to galactic background starlight with a temperature of \SI{5000}{\kelvin} and a density
of \SI{1}{\eV \per \cubic \centi \metre}. The most important seed for IC emission, however, is the high-energy photons 
produced by the synchrotron emission within the nebula itself.
All photon fields are assumed to have uniform number density within the nebula. While certainly accurate for the CMB photons, more 
accurate modeling of the photons' spatial distributions might improve results.
This SSC model has six free parameters, of which five describe the shape of the
electron distribution $A$, $E_{\text{min}}$, $E_{\text{max}}$, $\alpha$, $\beta$, and one describes the magnetic field strength $B$.
The flux points to which this model is fitted was recorded by 6 different telescopes. The hard X-Ray and soft gamma-ray fluxes up to ${\approx}\SI{E-5}{TeV}$ were 
observed by the INTEGRAL satellite and the SPI instrument onboard the Comptel satellite.
Fluxes from \SIrange{E-4}{E-1}{\TeV} were recorded by the \fermi satellite.
Above ${\approx}\SI{E-1}{TeV}$ ground-based IACTs measure the highest gamma-ray energies. 

The synchrotron and IC emission was calculated using the numerical approximations implemented in the 
\naima~\cite{naima} package. Fitting was performed using Markov-Chain Monte-Carlo sampling (MCMC) 
on the posterior using the \emcee~\cite{emcee} sampler. 
By default, \naima uses a Gaussian likelihood assuming independent measurement errors on the data
\begin{eqnarray}
  \operatorname{\mathcal{L}}\left(\mathbf{F} \mid \mathbf{p} \right) = \prod_i^N \operatorname{\mathcal{N}}(\mathbf{F}_i \mid \mu\! =\! \operatorname{SSC}(\mathbf{p}), \, \sigma\! =\! \sigma_i),
\end{eqnarray}
where $\mathbf{F}$ is a vector of flux measurements with corresponding uncertainties $\sigma$ and $\mathbf{p}$ is the parameter vector
for the SSC model.
The independence assumption for the flux errors is almost certainly not correct. It is, however, a pragmatic approach when 
working with flux data which is often published without further information about possible correlations between the points. 
All priors were assumed to be uniform, or uniform in logarithmic space.

Even though implementation of radiative models in \naima is relatively efficient, evaluating the model on hundreds of thousands of Markov chain samples 
takes hours. To speed up the sampling, I have built a lookup table of values of the SSC model evaluated on a grid of \num{1000000} parameter combinations.
Building the lookup table takes several hours on a large machine with 24 CPU cores.
The samplers then evaluated the model using linear interpolation between the grid points. Sampler settings can then be adapted 
and tuned without recalculating the lookup table. For the final fit a total of \input{build/naima_results/num_samples.txt} samples 
where taken in \input{build/naima_results/num_chains.txt} parallel chains.
The resulting fit values are calculated from the median of the marginalized posterior distributions. The provided errors 
are taken from the 16\th and 84\th percentile.

\begin{center}
  \begin{tabular}{l@{\hskip 0.5em} r@{\hskip 0.5em} l@{\hskip 3em}l@{\hskip 0.5em} r@{\hskip 0.5em} r}
    \text{\input{build/naima_results/name_0.txt}} &=&$ \input{build/naima_results/param_0_raw.txt}$ &\text{\input{build/naima_results/name_1.txt}} &=&$\input{build/naima_results/param_1_raw.txt}$\\
    \addlinespace[0.5em]
    \text{\input{build/naima_results/name_3.txt}} &=&$ \input{build/naima_results/param_3_raw.txt}$ &\text{\input{build/naima_results/name_2.txt}} &=&$\input{build/naima_results/param_2_raw.txt}$\\
    \addlinespace[0.5em]
    \text{\input{build/naima_results/name_4.txt}} &=&$ \input{build/naima_results/param_4_raw.txt}$ &\text{\input{build/naima_results/name_5.txt}} &=&$\input{build/naima_results/param_5_raw.txt}$
  \end{tabular}  
\end{center}

\Cref{fig:ssc_fit} shows the fitted model together with the measured flux data. 
Even though this model uses relatively simple physical assumptions, it accurately represents the observed fluxes in the high-energy end of the SED.
\Cref{fig:model_variations} shows the influence of each single parameter on the shape of the model spectrum.
\Cref{tab:ssc_fit_results} in the appendix shows the values of the fitted parameters together with 
images of the marginalized posterior distributions and the sampled chains.
\Cref{fig:ssc_correlation} in the appendix shows the correlation between each of the fitted parameters.
More details on MCMC methods in general, will be given in \cref{sec:mcmc}.
The code to reproduce these results is available at 
\githubcenter{kbruegge/simple_ssc_model} 
% TODO. priors in table?
% TODO. MCMC section? man so much to write still?
\begin{figure}
  \centering
  \input{build/ssc_fit.pgf}
  \caption[Fit of SSC model to Crab Nebula data]{The full SSC (\ssclong) model plotted together with the observed data.
  The colored error bars show the observed fluxes by the six different instruments.
  The black line shows the median of the values sampled by the Markov chain. The error band around the black line is built from randomly chosen samples in the chain. 
  For each sampled parameter set, the SSC model is drawn as a gray transparent line. 
  Burn-in samples have been discarded before producing this plot. Despite the 
  simple model assumptions, the SED accurately matches the data in the VHE gamma-ray end.}
  \label{fig:ssc_fit}
\end{figure}

\begin{figure}[p]
  \centering
  \includegraphics[width=\textwidth]{build/model_variations.pdf}
  \caption[SSC model with varying parameters]{The full SSC model plotted together with measured fluxes. In each image one of the parameters 
  of the model is varied while the others remain fixed. As stated in the text, this model assumes one population of high-energy electrons 
  which are distributed according to a log-parabolic energy spectrum with high and low energy cutoffs. Each of the six 
  free parameters is varied as indicated by the colorbars.}
  \label{fig:model_variations}
\end{figure}


