%%%%%%%%%%%%%%%%%%%%%%%%%%%%%%%%%%%%%%%%%%%%%%%%%%%%%%%%%%%%%%%%%%%%%%%%
\chapter{Introduction}
%%%%%%%%%%%%%%%%%%%%%%%%%%%%%%%%%%%%%%%%%%%%%%%%%%%%%%%%%%%%%%%%%%%%%%%%

% Since the early days of the last century, scientists have been searching for the sources of cosmic radiation. 
The discovery of cosmic rays during the daring balloon flights of Victor Franz Hess in 1912~\cite{hess_original} opened up
an entirely new window into the universe.
This elusive radiation, which so relentlessly bombards us from outer space, carries a wealth of information
about the most violent processes in the cosmos. 
Probing the gamma-ray sky is crucial to understanding the processes which drive the cosmic-ray acceleration.
While charged cosmic rays are deflected by magnetic fields, gamma rays pinpoint back to the source and allow us to image these objects. 
Only the combination of data from multiple facilities can help to unravel the inner workings of
cosmic-ray sources. 
The success of these joint campaigns has become apparent recently through the first observational evidence
of neutrino emission from the blazar TXS 0506+056~\cite{icecube2018multimessenger}. 
This collective effort used data from the IceCube neutrino observatory in Antarctica as well as data from the gamma-ray 
experiments \fermi and \magic. 
From its very beginning, the \fermi collaboration made all of its recorded data and software available to the general public.
With the recent advent of multi-messenger astronomy it has become critical for imaging atmospheric Cherenkov telescopes (IACT)
to openly share their data as well.
This first part of my thesis deals mainly with the
computation of energy spectra and flux point estimates in an open and reproducible manner and tries to motivate the need for public IACT data. 
The text aims to convey all the essential information that is needed in order to understand the measurement process of IACTs. 
% The following chapters serve as a single comprehensive 
% The following chapters target the uninitiated reader 

The galactic cosmic ray population is seeded by remnants of cataclysmic supernova events.
The Crab Nebula is the prototypical object of this kind.  
It is a steady source of bright \si{TeV} gamma-ray emission in the northern sky and is 
continuously observed by radio, X-ray,  and gamma-ray observatories. 
It was the first gamma-ray source detected by an imaging atmospheric Cherenkov telescope (IACT) in 1989~\cite{whipple_crab}. 
In \cref{ch:theory} I describe the typical emission processes prevalent in supernova remnants (SNR) 
and describe how log-parabolic energy spectra emerge in many sources of cosmic rays.
\Cref{ch:observatories} gives an overview of experimental techniques used to observe gamma rays. 
In \cref{ch:crab-sed} I show that the spectral energy distribution of the Crab Nebula,  in an energy range from 
a few \si{keV} to tens of \si{TeV}, can be described with a single electron population of log-parabolic shape.
In that chapter I model the synchrotron, inverse Compton, and \ssclong emission using the \naima~\cite{naima} program
and fitted the model to data from six different experiments using Markov chain sampling. 
This emphasizes the importance of open data in the Cherenkov astronomy community.
Without open access to flux data from multiple wavelengths, no model assumptions can be validated. 
% The ultimate goal my thesis will be to propose a reproducible and open-source analysis chain 
% for IACTs.

\Cref{ch:spectral} goes through all the harrowing details needed in order to understand the
measurement process of IACTs, a critical part of which is the 
computation of the instrument response functions. In \cref{sec:irf} these response functions are
formalized in an instrument-agnostic way, 
the ultimate goal of which is the construction of a common data format that can be used by any IACT instrument. 
Traditionally, all IACT collaborations use their own proprietary software and data formats to produce high-level results like flux points. 
In a mutual endeavor between the \magic, \hess, \veritas, and \fact collaborations to change the current state of affairs,
a small data sample of Crab Nebula observations was made public in this common data format. 
Using this data we published the first joint and fully open-source analysis of combined IACT data~\cite{joint_crab}.
In \cref{sec:spectral_fit} I use the published event lists and instrument response functions
to fit a log-parabolic spectral model to the IACT data using Hamilton Markov chain sampling. 
\Cref{sec:unfolding} shows how that same data can be used to unfold the flux points for each individual instrument.


