
\chapter{Observation of Very-High-Energy Gamma Rays}
\label{ch:observatories}
Cosmic rays have been observed using space-born as well as ground-based scintillation detectors for several decades. The
highest-energy cosmic rays have been recorded by the AUGER instrument reaching up to \SI{e20}{\eV} or \SI{16}{\joule} of
kinetic energy. While their isotropic arrival directions indicate their extragalactic origin, no point-source of cosmic rays
could be identified as of today~\cite{auger_isotropic}.
Measuring gamma-rays or neutrinos presents one useful advantage over observations of electrically charged cosmic rays.
Photons and neutrinos are oblivious to electric and magnetic fields in space and travel in a straight line from their source
to the observer. This makes them prime candidates for learning more about the processes and objects in which
they are created. Neutrinos, however, are notoriously hard to capture due to their low mass and neutral charge. They only interact through the weak
force requiring large detection volumes. So far no extragalactic point source of neutrinos could be identified though strong hints indicate
they are produced in active galactic nuclei~\cite{icecube2018multimessenger}.
The observation of high-energy gamma rays  requires much lower detector volumes and allows for observations into energy ranges of many \si{TeV}.
Since the first successful gamma-ray astronomy missions in the early 1960s, the improvement of space-faring technology and sensor equipment
now allows us to identify thousands of distinct sources of high-energy gamma radiation.
% nach oben tun
% The most prominent sources being active galactic nuclei or supernova remnants and pulsars within our own galaxy. % TODO maybe remove

\section{Satellite Experiments}
The first gamma-ray telescope was launched into space onboard the Explorer XI mission in 1961~\cite{explorer}.
It carried a sandwiched scintillator with an area of only \SI{45}{cm^2} and detected about
\num{1000} gamma-rays during its 7-months-long mission. While not being able to pinpoint any sources, it is considered to
be the first measurement of gamma rays of cosmic origin. Gamma-ray bursts (GRBs) were serendipitously
discovered by satellites of the Vela mission in 1967, which were originally built to monitor the atmosphere for nuclear blasts~\cite{vela}.
More gamma-ray missions followed. Among them the Compton Gamma-Ray Observatory (CGRO), Swift, AGILE and the Integral mission. % todo. search references
The most recent mission, the Fermi Gamma-ray Space Telescope, was launched in 2008 and put into orbit about \SI{550}{km} above Earth's surface.
It is equipped with two detectors, the Gamma-ray Burst Monitor (GBM) and the Large Area Telescope (LAT).
The GBM's primary purpose is to detect gamma-ray bursts within a large field of view. It can detect photons
with energies between \SI{8}{keV} and \SI{40}{MeV}. As the name suggests, it was built to detect short bursts
of gamma radiation in the sky. The bursts can last anywhere from minutes to seconds and are among the most powerful and bright events
ever seen in the sky. About 200 to 300 GRBs are detected per year by the GBM~\cite{fermi_gbm}.
The LAT detector onboard FERMI covers an energy range of \SI{20}{MeV} to \SI{300}{GeV} with a field of view of
\SI{2.4}{sr} covering about \SI{20}{\percent} of the sky any one time.
It was designed to catalog and monitor several thousand sources of gamma rays and record the energy spectra~\cite{fermi_lat}.
During its active years it has become an invaluable instrument for the cataloging and discovery of gamma-ray sources.
The most recently published catalog~\cite{4fgl} contains more than \num{5000} sources. Figure~\ref{fig:4fgl} shows a map
of all sources found in the 4FGL catalog along with their gamma-ray flux between \SI{100}{MeV} and \SI{100}{GeV}.

\begin{figure}[h]
  \centering
  \includegraphics[width=\textwidth]{build/4fgl.pdf}
  \caption[Skymap of Fermi's 4FGL source catalog]{A sky map depicting each distinct source found in the 4FGL catalog combining data measured by the
  Fermi-LAT instrument over the course of 8 years. This is a Mollweide projection in galactic coordinates clearly showing
  a clustering of sources around the galactic plane. The colors indicate the total flux emitted by each source
  in an energy range of \SIrange{0.1}{100}{GeV}.}
  \label{fig:4fgl}
\end{figure}


\section{Imaging Atmospheric Cherenkov Telescopes}
\label{sec:iact}
Imaging Atmospheric Cherenkov Telescopes (IACTs) use the atmosphere as their detection medium. 
Each cosmic ray or gamma ray hitting Earth's atmosphere interacts with the nuclei in the surrounding 
gas. At sufficient energies, this interaction will create secondary particles, which in turn interact
with the surrounding medium. 
This process kicks off a cascade of particles moving towards the surface. 
This so-called \emph{air shower} keeps growing until the particles' energies are insufficient to produce new 
offspring. Very energetic primary particles can induce cascades that reach the Earth's surface. 
The interactions governing the air shower induced by a primary hadron are severely different from 
those induced by a primary gamma ray or electron.
An incoming electron or gamma ray interacts mainly through bremsstrahlung and pair production. 
This type of air shower is of purely electromagnetic nature. An incoming electron radiates a 
gamma-ray through bremsstrahlung in the presence of a nucleus. The new photon emerging from this 
\emph{collision} either reaches the ground and gets absorbed or, given enough energy, 
creates a new electron-positron pair. 
This process continues until no new particles can be formed below an energy of roughly \SI{1}{MeV}.
A cosmic hadron, i.e. a proton or heavier nucleus, also starts a process of successive collisions. 
The number of sub-particles created in each collision depends on the parent particles' energy~\cite[78]{grieder_air_shower}. 
\autoref{eq:hadron_collision} represents a proton-nucleon interaction, where $N$ is the target nucleus in the atmosphere
and $X$ represents some remaining fragments of $N$. 
\begin{equation}
  \label{eq:hadron_collision}
  \Pproton + N \rightarrow \Pproton + X + \Pgp^{\pm,0} + \PK^{\pm,0}\ldots\text{\cite{kampert_cosmic}}
\end{equation}
The process will propagate through the atmosphere until no more sub-particles can be created below the rest energy of the pion 
near \SI{140}{MeV}.
Each hadronic air shower also has an electromagnetic sub-shower due to the gamma rays produced in pion decay
\begin{equation}
  \label{eq:pion_decay}
  \Pgpz \rightarrow 2 \gamma.
\end{equation}
The air showers produced from hadronic primaries can be distinguished from the purely electromagnetic counterparts 
through various observables. Most important is the \emph{lateral} spread of the shower and the 
time development of the number of particles present in the shower, both of which can be observed by ground-based telescopes.
The charged component of an air shower produces flashes of Cherenkov light and makes it possible to \emph{take an image}
of the shower as its moving through the atmosphere. 
Early on, scientists working with radioactive material noticed a faint blue glow in water near strong radioactive sources~\cite[835]{grieder_air_shower}. 
Pavel Cherenkov began studying it systematically in 1934. He later shared the Nobel price for its discovery with Frank and Tamm in 1958~\cite{nobel_1958}. 
Cherenkov light is emitted by charged particles moving through a medium at superluminal speeds. 
While the speed of light in vacuum $c_0$ is constant in all reference frames, the speed of light in 
transparent media is slower. The speed of light in a material is expressed in terms of its 
refraction index $n=\nicefrac{c_0}{v}$. The charged high-energy particles present in an air shower can 
have velocities higher than the speed of light in air. If they do, they radiate Cherenkov light.
This bluish light is emitted along the direction of the moving particle.
The opening angle of the Cherenkov light cone generated by a single charged particle of velocity $v$ is given as 
\begin{equation}
  \label{eq:opening_angle}
  \theta = \arccos\left(\frac{1}{\beta n(h)}\right)
\end{equation}
where $\beta = \nicefrac{v}{c_0}$ and $n(h)$ is the refractive index of air at height  $h$ above sea level.
As the refraction index $n(h)$ increases with pressure, the opening angle of Cherenkov light decreases at high altitudes. 
At an altitude of \SI{10}{\kilo\metre}, the opening angle will be close to \SI{0.8}{\degree}. At sea level the angle is closer to \SI{1.35}{\degree}.
Hence, the Cherenkov light is strongly focused along the trajectory of the moving charge. 
%\todo{photon distribtuion on the ground. read soem corsika file using eventio.}
A typical light flash produced by an air shower glows for approximately \SIrange{20}{30}{\nano\second}.
Capturing this faint and fast glimpse of an air shower requires sensitive instruments.
Cherenkov telescopes require purpose-built cameras with single-photon resolution and fast readout systems to 
image air showers.   
The first detections of a point-source of cosmic gamma radiation was performed with the Whipple telescope in Arizona in 1989~\cite{whipple_crab}.
It was the first detection of the Crab Nebula in the \si{TeV} range of light.
The next generation of IACT experiments followed promptly with the \hegra~\cite{hegra-crab-data}, \magic~\cite{magic}, \hess~\cite{hess}, \veritas~\cite{veritas} 
and \fact~\cite{fact} projects, of which all but \hegra are still operating.
Before the first successful observation of the Crab Nebula, many challenges and problems inherent to IACTs had to be overcome. 
The first IACTs struggled to differentiate between showers induced by cosmic rays from those induced by gamma rays. 
Even for very bright sources of gamma rays, the amount of cosmic rays triggering 
the telescope is many orders of magnitude higher than the desired signal. Cosmic rays effectively act as the major 
source of background noise in IACT data. 
Development of methods to perform effective background suppression took extensive work on simulations of air showers. 
These simulations are necessary since there is no artificial source of gamma rays or protons 
in the energy ranges probed by IACTs. Experiments sensitive to lower energies can be calibrated in a laboratory setting, where their response 
to incoming particles can be measured in great detail. For IACTs simulations are the only way to gauge the instrument's response.
Today, background suppression, or gamma-hadron separation, is performed using machine-learning algorithms which have been trained 
on simulations. \Cref{ch:ml} goes into more detail about the algorithms used.
By today, the success of Cherenkov astronomy is self-evident given its huge 
contribution to our understanding of active galactic nuclei, supernova remnants and the gamma-ray sky in general. One recent result from the 
\hess collaboration~\cite{hess_gps} is displayed in figure~\ref{fig:hgps}. It shows a gamma-ray view of the galactic plane 
in energies above \SI{1}{TeV}. 
% In chapter~\ref{sec:spectral} I will go into more detail about the \fact, \magic, \hess, and \veritas telescopes. 
% Building an IACT always involves a trade-off between its instrumented volume and sensitivity to lower energies. 
% Since larger mirrors collect more Cherenkov light, fainter showers can be recorded. The field-of-view of 
\Cref{ch:cta} contains details on the next-generation Cherenkov Telescope Array (CTA) project and its data analysis.

The following sub-sections describe the four currently operating Cherenkov telescopes and the astronomer's apparent fondness for intricate 
abbreviations.

\subsection{\magic}
The Major Atmospheric Gamma Imaging Cherenkov Telescopes~\cite{magic}, or \magic, are a pair of Cherenkov telescopes located on the Canary island of La Palma off the west coast of Africa.
It is part of the Roque des los Muchachos observatory on the island's vulcanic remnant at a height of approximately \SI{2200}{\metre} above sea level.
The two telescopes feature large segmented mirrors with a total diameter of \SI{17}{\metre} per telescope with a field of view of \SI{3.5}{\degree}. 
The first of the two telescopes was operated in monoscopic mode from 2004 to 2009 until the second telescope was ready for operations. 
\magic can detect relatively faint sources due to its large mirrors. 

\subsection{\veritas}
\veritas, the Very Energetic Radiation Imaging Telescope Array System~\cite{veritas}, is an array of 
four telescopes. Each telescope has a mirror with a diameter of \SI{12}{\metre} and a field of view of \SI{3.5}{\degree}. The \veritas 
telescopes are located at the Fred Lawrence Whipple Observatory, Arizona, in just about 3 hours driving distance from the state's capital Phoenix.
The four telescopes are located on the corners of a rectangle approximately \SI{100}{\metre} apart from each other. \veritas observes both active galactic nuclei 
as well as sources in our own galaxy.  

\subsection{\fact}
The First G-APD Cherenkov Telescope (\fact) was the first of its kind to employ silicon photo-multipliers for VHE gamma-ray astronomy. 
It is located next to the \magic telescopes and shares much of its infrastructure with it. \fact is a single IACT with a small mirror with a diamter of \SI{4}{\metre} 
and a field of view of \SI{4.5}{\degree}.
FACT is dedicated to observe and monitor bright active galactic nuclei on the northern sky. \fact is fully autonomous and does not require operators
on site. It observes a list of predefined sources at night and automatically contacts a remote shifter via a phone call in case any problems 
appear during observation~\cite{fact_robotic}. 

\subsection{\hess}
The High Energy Stereoscopic System~\cite{hess}, \hess, is the largest of all currently operating Cherenkov telescopes and the only one operating 
south of the equator.
It consists of 5 telescopes situated at an altitude of \SI{1800}{\metre} in the Namibian highlands.  
Its location in the southern hemisphere allows for long-term observations of the center of the Milky Way as seen in \cref{fig:hgps}.
The largest of \hess's telescopes has a diameter of \SI{28}{\metre} covering an area of about \SI{615}{\square\metre}.
The sizable aperture reduces the depth of field considerably. To adjust the focus of the optical system, 
the camera can be moved along the optical axis of the telescope electronically.   
The four smaller telescopes each have a diameter of \SI{13}{\metre}. These four telescopes are arranged in a square
with an edge length of \SI{120}{\metre} with the large telescope placed in its center.
The first phase of \hess operations began in 2004. The fifth telescope was added in 2012, reducing 
the energy threshold of \hess to several tens of \si{GeV}. 
A unique design feature of \hess is its ability to quickly dismount the cameras from the telescopes. This simplifies 
maintenance works and hardware upgrades.

\begin{figure}
  \centering
  \includegraphics{build/hgps.pdf}
  \caption[The \hess galactic plane survey]{The \hess galactic plane survey. \hess measured the diffuse gamma-ray flux in the galactic plane over several years and published the
  data in 2018~\cite{hess_gps}. The colors show the integral flux emitted in energies above \SI{1}{TeV}~\cite[\S4.3.2]{hess_gps}.
  The figure shows a region near the galactic plane in longitudes \protect{\input{build/hgps_lon_range.txt}}
  and latitudes \protect{\input{build/hgps_lat_range.txt}}. On the left-hand side of the upper panel, a supernova remnant
  RX J0852.0-4622 is clearly recognizable due to its shell-like structure.
  Sagittarius A in the center of our galaxy is marked in the third panel.}
  \label{fig:hgps}
\end{figure}

\subsection{CTA}

The Cherenkov Telescope Array (CTA) will be the largest of all earth-bound gamma-ray observatories. 
It is currently in its planning stage with the first prototype telescopes under construction. 
Its current design includes over 100 telescopes that will be stationed at La Palma and the Paranal observatory in Chile.
More details about CTA will follow in \cref{ch:cta}. Suffice it to say at this point, it will be the largest array of optical telescopes that 
has ever been operated. With an estimated cost of over \texteuro\,\num{300} million~\cite{cta_website}, it will also be the
most expensive operation in the history of ground-based astroparticle physics.

\section{Water-Cherenkov Experiments}
The particles produced in air showers can be captured using scintillation water tanks on the surface.
A dense spacing of tanks and light sensors helps to measure details about the shower structure, which can help to differentiate
between air showers induced by hadrons and air showers induced by photons.
The High Altitude Water Cherenkov Experiment (HAWC)~\cite{hawc} is the latest gamma-ray observatory using
water tanks. It is located in Mexico at an altitude of \SI{4100}{\metre} and has been operating since 2015.
HAWC consists of 300 water tanks with a total water content of 56 million liters. Each tank is fitted with 4 photomultiplier tubes.
The charged component of air showers produces Cherenkov light in the tanks. The arrival time of the shower front in the tanks
is the prime indicator for the direction of the incident shower. This method of water Cherenkov detection makes it easy
to fill relatively large detector volumes hence making HAWC more sensitive to high gamma-ray energies beyond
the capabilities of the FERMI satellite. 
HAWC's detection capability peaks near \SI{10}{TeV} of gamma-ray energy.
The absorption of high-energy photons due to interaction with extragalactic background light reduces the maximum distance at which HAWC can detect sources.
Thus, HAWC is best suited for observing bright sources within our own galaxy.
One considerable advantage of HAWC is its large field of view and its capability to operate during daytime.

