
\chapter{Notes on Reproducibility}
\label{ch:repro}

The computation of the optimized event selection is the final link in a long chain of analysis steps. 
Many terabytes of raw data are preprocessed into tabular data on which machine-learning algorithms can be trained. 
The data needs to be split into independent test and training sets to avoid biases. All meta data concerning the air-shower
simulation has to be carried along so that the effective area can be calculated for any arbitrary subset of the data. 
The entire process is a composition of many different programs which are in principle self-contained. 
Without proper automatization, the whole construction is rather fragile, if not dangerously prone to error.

Each figure and table in this document has an implicit dependency on some input data.
\Cref{tab:event_selection}, for example, shows the results of the optimized event selection.
The table can only be constructed if the machine-learning models described in \cref{ch:ml} have been applied to the test data 
which resulted from the preprocessing performed in \cref{ch:cta_analysis}. 
The models themselves in turn depend on the configuration files and the training data. 
These data dependencies can be explicitly modeled with tools such as 
\make~\cite{make} or \snakemake~\cite{snakemake}.
I chose \make because it is supported on essentially every operating system that is in use today.   
Data dependencies are described by so-called \emph{Makefiles}. The \make program builds a directed, acyclic graph from 
the Makefile and executes each step in topological order. 
% The first call to \make executes the entire graph so that all defined outputs are generated.
% Subsequent calls to \make only rebuild the sub-graph which depends on the modified input.
% Originally build to speed up the compilation of large software projects, its ability to model data dependencies in 
% a single text file is useful in a more general context.
This entire document can be build from a single call to the \make program. Each machine-learning model, fit, figure, table, and automated 
\LaTeX snippet is part of the execution graph. 
\Cref{fig:dep_graph} in the appendix shows a visual representation of the dependency graph for this document.

All software used for this thesis was written in the \python programming language in version 3.7.2. Each plot was created using the 
\matplotlib~\cite{matplotlib} library version 3.1.0. A full list of \python dependencies can be found in the \texttt{requirements.txt} file that is 
attached in \cref{sec:requirements}.
The \LaTeX code for this document and all scripts needed to create the figures and tables will be uploaded after official publication to 
\githubcenter{kbruegge/phd-thesis}

