\chapter{Acceleration of Cosmic Rays and Gamma Rays}
\label{ch:theory}
Earth's atmosphere is constantly being bombarded by very-high-energy particles of cosmic origin.
Among these are photons, neutrinos, electrons, protons and heavier nuclei.
The astroparticle community usually refers to the hadronic particles as \emph{cosmic rays}, while neutrinos and high-energy photons
are viewed as separate entities.
The discovery of cosmic rays and their byproducts in the atmosphere is attributed to the balloon experiments performed by Victor
Hess in the year 1912~\cite{hess_original, hess_nytimes}.
The term \emph{cosmic rays} was coined by robert A. Millikan 19 years later~\cite{millikan} after performing his own
observations of ionizing radiation at several different altitudes.
Cosmic rays cover a vast range of energy spanning 15 orders of magnitude from a few \si{\kilo\eV} to several \si{EeV}.
One major goal in the field of cosmic-ray astronomy is to learn more about the origin of these cosmic messengers.
The key mechanism that drives cosmic rays is believed to be shock acceleration.
Satellites have gathered direct evidence of particle
acceleration in the \si{\kilo\eV} to \si{\mega\eV} range in interplanetary shocks in our solar system.
Even in our direct neighborhood, Earth's bow shock with the surrounding medium, particles are accelerated to higher energies~\cite{bow_shock}.
Interactions in the hot and dense plasma ejected by solar flares accelerate particles up to \si{GeV} energies~\cite[236]{gaisser}.
Beyond \si{\giga\eV} energies, however, the engines for cosmic particle acceleration lie outside of our solar system.
The population of galactic cosmic rays is most likely driven by shock acceleration in supernovae remnants
within the Milky Way.
At even higher energies, beyond \si{EeV}, galactic sources cannot account for the observed fluxes. Protons at these energies
are not confined by the magnetic fields in our galaxy and can escape into the intergalactic regions.
The confinement of a particle is governed by its radius of gyration.   
A particle with rest mass $m_0$ and Lorentz factor $\gamma$ has kinetic energy $E_{\text{kin}}$
\begin{equation*}
  E_{\text{kin}} = (\gamma - 1) m_0 c^2.
\end{equation*}
Given the cosmic ray energy of \SI{1}{EeV} and rearranging for $\gamma$ yields
\begin{equation*}
  \gamma = \frac{E_{\text{kin}}}{ m_0 c^2}  +  1 \approx \num{1.1E9}.
\end{equation*}
The gyradius radius $r$ of a charged particle with charge $q$ which is moving at velocity $v$ perpendicular to a magnetic field of strength $B$ is
\begin{equation*}
  r = \frac {p}{ q B} = \frac{\gamma m v}{q B}.
\end{equation*}
Assuming a mean magnetic field of $B = \SI{5E-10}{T} $ within the Milky Way~\cite{haverkorn_magnetic}, results
in a gyration radius of approximately $r = \SI{5e18}{\metre} = \SI{528}{\lightyear}$. At these energies particles can escape
the local arm of the galaxy. Hence, particles with higher energies most likely originate in extragalactic sources.
Jets of active galactic nuclei (AGN) are widely accepted to be the source of these particles.
Figure~\ref{fig:cosmic_rays} shows the differential energy spectrum of cosmic rays.
By multiplying the flux with $E^{2.7}$ the steep power law becomes flat. The first bend in the spectrum around
\SI{1E3}{TeV} is often called the \emph{knee}. The second bend is called the \emph{ankle} near
\SI{3E6}{TeV} and its often associated with the appearance of extra-galactic particles. The shape of the spectrum
between the knee and the ankle is still matter of some debate. The steep cutoff visible near \SI{5E7}{TeV} is assumed to be due to the
Greisen–Zatsepin–Kuzmin (GZK) cutoff~\cite[209]{gaisser}. The origin and composition of cosmic rays beyond the cutoff is still unclear.
Shock acceleration can be modeled by a process called \emph{Fermi acceleration} and a \emph{plasma dynamic} approach.
Both approaches predict power-law like flux. They are briefly described in section \ref{sec:fermi}.

The production and acceleration of cosmic rays is strongly intertwined with the production of gamma rays and neutrinos.
Differential energy spectra, or flux curves,  show the energy resolved light emission of an astrophysical source.
In the gamma-ray community they are usually given in units of \si{cm^{-2}.s^{-1}.TeV^{-1}}. The flux is often scaled 
by the square of the photon energy $E^2$ for better visual representation leading to units of \si{cm^{-2}.s^{-1}.TeV}.
This scaled quantity is often called a spectral energy distribution (SED)~\cite[3]{gaisser}.
Both $\nu f_{\nu}$ and $E^2 \frac{\diff{N}}{\diff{E}}$ are common shorthands for these SEDs. I will use the latter designation
in this text.
SEDs are of major interest to many astronomers, as their shape can be used to validate or invalidate models of acceleration mechanisms in these sources.
SEDs often span more than 15 orders of magnitude in photon energy and combine a multitude of instrument technologies and 
disciplines. 
The gamma rays in the very-high-energy range of the SEDs, above \SI{100}{GeV}, are produced by either synchrotron radiation or inverse Compton (IC) scattering
on photon fields. Sections \ref{sec:synchro} and \ref{sec:ic} give a little more detail about the shape of synchrotron and IC spectra
for supernova remnants.
Bright sources are extensively monitored by radio, infrared, optical, X-ray and gamma-ray telescopes.
The most prominent source of gamma radiation within our galaxy is the Crab Nebula.
It is a supernova remnant which exploded in 1054. The event was recorded by Chinese astronomers which reported a
\enquote{Guest Star} which had been visible for three weeks during daytime~\cite{crab_chimera}.
A scan of the original report by the astronomer and a translation can be found in \cref{ap:crab}.
The matter ejected during the explosion has since expanded into a shell of hot plasma with a radius of about 1.5 lightyears~\cite{crab_chimera}.
Extensive observations from in the radio, optical, X-Ray, and gamma-ray bands, have made it
one of the most popular objects for astronomers to study.
For very-high-energy (VHE) gamma-ray astronomers in particular, the Crab Nebula is of major interest due to its steady and bright emission of photons.
In Cherenkov astronomy it is common to test new data analysis techniques on data from \emph{the Crab}, as it is often called in the
community. The analysis I performed for this thesis is no exception.
\begin{figure}
  \centering
  \input{build/cosmic_rays.pgf}
  \caption[The all-particle cosmic-ray spectrum.]{All particle cosmic-ray spectrum measured by the Auger, IceTop, Tibet, Kascade, and HiRes experiments.
  The energies span 10 orders of magnitude and range from a few \si{\GeV} to some \SI{100}{EeV}. The plot only shows
  the high-energy end of the spectrum starting range from \SI{100}{\TeV}.
  Cosmic rays can reach energies equivalent of a baseball flying at \SI[per-mode=fraction,fraction-function=\nicefrac]{50}{\kilo\metre\per\hour}.
  The plotted flux is multiplied by $E^{2.7}$ to highlight the structural features of the spectrum. The \emph{knee} and \emph{ankle}
  are visible at approximately \SI{1}{PeV} and \SI{5}{EeV}.}
  \label{fig:cosmic_rays}
\end{figure}
\newpage
\section{Fermi Acceleration}
\label{sec:fermi}
In 1949 Enrico Fermi published a model to explain the high energies observed by cosmic-ray detectors and the power-law shape of the cosmic-ray spectra~\cite{fermi_original}.
Prior to Fermi's publication the origin of the cosmic rays at the highest energies was unclear. 
Fermi summarized the situation as follows:
\begin{displayquote}
  The argument against the conventional view that cosmic radiation may extend at least to all the galactic 
  space is the very large amount of energy that should be present in form of cosmic radiation if it were to extend to such a huge space.
  Indeed, if this were the case, the mechanism of acceleration of the cosmic radiation should be extremely efficient.
\end{displayquote}
The principal idea published by Fermi is that particles, on average, gain energy by \emph{collisions} with randomly moving magnetic fields.
These \emph{magnetic mirrors} were believed to be clouds of plasma moving at high velocities in random directions.
As to the origin of these fields, no explanation was given.
The power-law energy spectrum for this type of particle acceleration follows from stochastic arguments~\cite[\S12.2.1]{gaisser}.
Assume some test particle gains energy $\Delta \gamma = \alpha \gamma$ in each collision with the magnetic mirror.
After $n$ collisions the energy will be
\begin{align*}
  \gamma_n &= \gamma_{n-1} + \gamma_{n-1} \alpha \\
  \gamma_n &= \gamma_{n-1}(1 + \alpha) \\
  \gamma_n &= \gamma_{n-2}(1 + \alpha)(1 + \alpha) \\
  &\setbox0\hbox{=}\mathrel{\makebox[\wd0]{\hfil\vdots\hfil}} \\ %see https://tex.stackexchange.com/questions/7650/centering-vdots-in-a-system-of-many-equations
  \gamma_n &= \gamma_0 (1 + \alpha)^n,
\end{align*}
where $\gamma_0$ is the initial energy of the particle. Solving this equation for $n$ yields
\begin{equation*}
    n = \frac{\log\left(\frac{\gamma_n}{\gamma_0}\right)}{\log(1 + \alpha)}.
\end{equation*}
In this scenario, the particle keeps bouncing around between the moving magnetic fields until it escapes the acceleration region.
Let $P_{\text{esc}}$ be the constant probability for a particle to escape its confinement at any given time.
Then,  after $n$ collisions, the probability for the particle to remain within the acceleration region is  $P_n = (1 - P_{\text{esc}})^n$.
Substituting $n$ and using the identity $x^{\log(y)} = y^{\log(x)} $ results in
\begin{align*}
  P_n &= (1 - P_{\text{esc}})^n \\
      &= (1 - P_{\text{esc}})^{\frac{\log\left(\frac{\gamma_n}{\gamma_0}\right)}{\log(1 + \alpha)}} \\
      &= \left((1 - P_{\text{esc}})^{\log\left(\frac{\gamma_n}{\gamma_0}\right)}\right)^{\frac{1}{\log(1 + \alpha)}}\\
      &= \left(\frac{\gamma_n}{\gamma_0}\right)^{\frac{\log(1 - P_{\text{esc}})}{\log(1 + \alpha)}}\\
      &= \left(\frac{\gamma_n}{\gamma_0}\right)^{-s},
\end{align*}
where $s = \nicefrac{\log\left(\frac{1}{1 - P_{\text{esc}}}\right)}{\log(1 + \alpha)}$.
This acceleration model results in an energy spectrum which follows the power-law distribution
so ubiquitous in astroparticle physics. Fermi's model was able to successfully explain the experimental data recorded 
by cosmic-ray researchers at the time.

Another approach to the problem, which is perhaps a little more motivated by physics rather than statistics, can be build from the so-called \emph{diffusion-loss} equation. 
In this explanation, the magnetic fields described by Fermi are due to turbulent plasmas giving rise to
\enquote{random} movements and strong magnetic field gradients.
Following the notation in~\cite[566]{longair} the diffusion equation is defined as 
\begin{equation}
  \label{eq:diffusion-loss}
  \frac{\diff{N}(\gamma)}{\diff{t}} = D \nabla^2 N(\gamma) - \frac{\diff{}}{\diff{\gamma}}\bigl(\dot{\gamma} N(\gamma)\bigr) - \frac{N}{t_{\text{esc}}} + Q(\gamma),
\end{equation}
where $D$ is the diffusion coefficient, $Q(\gamma)$ is a particle source term and $t_{\text{esc}}$ is the typical escape time for a particle.
Fermi's approach assumes a steady-state configuration without diffusion or source terms. So $Q(\gamma) = 0$ and  $D \nabla^2 N = 0$. 
As seen above, the energy-gain term in the Fermi approach is postulated as $ \frac{\diff{\gamma}}{\diff{t}} = \alpha \gamma$.
This results in the simplified equation
\begin{align*}
  -\frac{\diff{}}{\diff{\gamma}} \left( \alpha \gamma N(\gamma) \right)  &= \frac{N}{t_{\text{esc}}}\\
  N(\gamma) + \gamma \frac{\diff{N(\gamma)}}{\diff{\gamma}} &= -\frac{N(\gamma)}{\alpha t_{\text{esc}}}\\
  \frac{\diff{N(\gamma)}}{\diff{\gamma}} &= -\left( 1 +  \frac{1}{\alpha t_{\text{esc}}} \right) \frac{N(\gamma)}{\gamma}.  \numberthis \label{eq:fermi_loss}
\end{align*}
\Cref{eq:fermi_loss} has a solution of the form 
\begin{equation}
  \label{eq:fermi_spectrum}
  N(\gamma) \propto  \gamma^{-\left( 1 +  \frac{1}{\alpha t_{\text{esc}}} \right)} \propto \gamma^{-s}.
\end{equation}
The diffusion equation results in a power-law spectrum just like the stochastic approach. 
The scenario Fermi described does introduce some problems, a discussion of which can be found in~\cite[566]{longair}. 
The energy gain per collision using the original Fermi model is proportional to the square of the magnetic 
field velocity $V$ i.e: $\Delta E \approx \beta^2 = \left(\nicefrac{V}{c}\right)^2$.
Hence, it is often called \emph{second-order} Fermi acceleration. This process alone is not efficient enough 
to explain the abundance of cosmic rays. The \emph{first-order} Fermi mechanism was discovered in the 1970s 
and takes place in the presence of strong shock waves~\cite[569]{longair}.
% TODO checkout cosmic ray propagation through magnetic turbulence to understand additional cutoffs. 
% dermer 12.5 and 12.2
Supernova remnants (SNR) provide the perfect conditions for shock wave acceleration i.e. first order Fermi acceleration. 
Exploding stars hurl massive amounts of matter into the space. The discarded shell expands rapidly into the
surrounding interstellar medium. A detailed discussion on acceleration in planar shock waves can be found in~\cite[\S12.2.2]{gaisser}.
Hence, supernova remnants like Cassiopeia A and the Crab Nebula are the perfect test objects to study models of cosmic-ray acceleration.


% TODO first vs second order see longair 17.3 and 17.4. maybe not needed after all.
% TODO spectral indices... also in longair 17.4 (page 572)

%\section{Pion Decay}
\section{Synchrotron Emission}
\label{sec:synchro}
A charged particle moving through a magnetic field radiates energy in form of light.
The breeding grounds of cosmic rays inevitably produce photons which can be observed on Earth. 
Hot gas and matter radiates thermal blackbody radiation. The largest part of the 
observed SEDs is produced by synchrotron radiation. It dominates the electromagnetic energy output over a broad range of wavelengths.
Suppose a single relativistic electron with Lorentz factor $\gamma$ moving in a magnetic field
produces a synchrotron radiation spectrum $F(\nu)$. Then an electron population with distribution 
$\nicefrac{\diff{N(\gamma)}}{\diff{\gamma}} \propto \gamma^{-s}$ leads to a radiated synchrotron spectrum of
\begin{align}
  \label{eq:single_el_synchrotron}
  \frac{\diff{N(\nu)}}{\diff{\nu}} &\propto \int \nicefrac{\diff{N(\gamma)}}{\diff{\gamma}} F(\nu) \diff{\gamma} \\
                                                &\propto \int \gamma^{-s} F(\nu) \diff{\gamma}.
\end{align}
The synchrotron spectrum of a single relativistic electron with Lorentz factor $\gamma$ moving in a magnetic field peaks 
strongly at 
\begin{equation}
  \label{eq:peak_synchro}
  \nu_s = \gamma^2 \nu_c = \gamma^2 \frac{B e}{2 \pi m_e}\; \text{\cite[284]{gaisser}.}
\end{equation}
If the power radiated per differential energy by a single electron is approximated by its peak frequency we get 
\begin{equation*}
  \frac{\diff{N(\nu})}{\diff{\nu}} \propto \int \gamma^{-s} \delta(\nu - \gamma^2 \nu_c) \diff{\gamma} \\
\end{equation*}
Substituting $x = \gamma^2 \nu_c$ and $\diff{\gamma} = 2 \gamma  \nu_c \diff{x} = 2 \left(\frac{x}{\nu_c}\right)^{1/2} \nu_c $ yields
\begin{align*}
  \frac{\diff{N(\nu)}}{\diff{\nu}} &\propto \int \gamma^{-s} \delta(\nu - x) \diff{x} \\
                                                &\propto \int \left(\frac{x}{\nu_c}\right)^{\frac{-s}{2}} \delta(\nu - x) \frac{1}{2 \left(\frac{x}{\nu_c}\right)^{1/2}} \diff{x} \\
                                                &\propto \int \left(\frac{x}{\nu_c}\right)^{\frac{-s - 1}{2}} \delta(\nu - x) \diff{x} \\
                                                &\propto \left(\frac{\nu}{\nu_c}\right)^{-\frac{s + 1}{2}}. \numberthis \label{eq:synchrotron_spectrum}
\end{align*}
The synchrotron spectrum emitted by the electrons will follow the electrons' power-law shape with a modified spectral index. 
The energy loss introduced to the original electron population due to synchrotron radiation changes the 
injected power-law spectrum. Suppose a Fermi process injects a power-law electron distribution
with spectral index $s$. 
Above some fixed break energy $\gamma_{\text{break}}$  the synchrotron loss steepens the electrons spectrum to $\gamma^{-(s+1)}$. 
The injected electron spectrum hence changes to 
\begin{equation}
  \label{eq:electron_break}
  \frac{\diff{N(\gamma)}}{\diff{\gamma}} \propto \begin{cases}
    \gamma^{-s} & \text{if $\gamma < \gamma_{\text{break}}$} \\ 
    \gamma^{-(s+1)} & \text{else.}
  \end{cases}
\end{equation}
This in turn changes the spectrum of the synchrotron emission as seen in \eqref{eq:synchrotron_spectrum}. 
Many SEDs of active galactic nuclei and supernova remnants show a very distinct synchrotron \emph{bump}. Measuring the 
synchrotron spectra gives direct evidence of the electron population in the source. The second major feature seen in many spectra 
is a consequence of the inverse Compton effect.
% \end{equation}
% We can estimate the effect of the synchrotron cooling on the spectrum by inserting \eqref{eq:injected_electrons} into the diffusion-loss 
% equation \eqref{eq:diffusion-loss}. Assuming no diffusion takes place, no additional sources of electrons exist, and the escape term 
% is negligible, the diffusion equation simplifies to
% \begin{equation}
%   \frac{\diff{N}(\gamma)}{\diff{t}} = -\frac{\diff{}}{\diff{\gamma}} \left( \dot{\gamma} N(\gamma) \right).
% \end{equation}
% The energy loss rate of a single electron due to synchrotron radiation is $\dot{\gamma} \propto \gamma^2 B^2$~\cite[284]{gaisser}
% which results in 
% \begin{equation}
%   \frac{\diff{N}(\gamma)}{\diff{t}} = -\frac{\diff{}}{\diff{\gamma}} \left( \dot{\gamma} N(\gamma) \right).
% \end{equation}
% todo check funk
% pages 219 longair
\section{Inverse Compton Emission}
\label{sec:ic}
Inverse Compton (IC) interaction is the driving mechanism for very-high-energy gamma rays detected from supernova remnants. 
Cherenkov telescopes almost exclusively observe the inverse Compton emission of these sources. 
The Compton effect describes the scattering of a photon with an electron. Given an electron at rest,
the incident photon will change its direction and wavelength during the scattering process. It is named after 
Arthur Compton, who was the first to publish a quantitative explanation of the effect in 1923. The Compton effect is of some
historical importance since its has no satisfactory explanation in a pure wave-like description of light. 
The wavelength shift of the scattered photon can only be explained with the particle nature of light.  
In the inverse Compton effect, the electrons are no longer considered to be at rest. A fast-moving electron transfers its energy to 
a photon during scattering. High-energy electron populations in SNRs can interact with any of the surrounding photon fields this way,
be it background starlight, infrared emission from dust clouds, or the cosmic microwave background (CMB).
The photons radiated via synchrotron emission can also seed the inverse Compton process. In sources like the Crab Nebula, this 
so-called \emph{\ssclong} (SSC) interaction is the key component in the VHE gamma-ray emission measured 
by Cherenkov telescopes.
% When approximating the frequency distributions of all these photon fields can roughly using their peak emission frequency 
% The blackbody radiation from thermal emission peaks at $\nu_{\text{wien}} \propto T$.
% The synchrotron emission for a single electron peaks at $\nu_s = \gamma^2 \nu_c$ . 
% The Compton process is defined by the Klein-Nishina cross section. It 
Approximating the seed photon fields as monochromatic, the shape of the inverse Compton spectrum roughly follows by a broken power law
as shown by~\cite{fouka_ic_shape, blumenthal_gould, lefa_kelner_shape} 
\begin{equation}
  \label{eq:ic_spectrum}
  \frac{\diff{N(\nu)}}{\diff{\nu}} \propto \begin{cases}
    \left(\frac{\nu}{\nu_c}\right)^{-\frac{s + 1}{2}} & \text{for $\nu h \ll m_e c$} \\ 
    \left(\frac{\nu}{\nu_c}\right)^{-(s + 1)} & \text{else}.
  \end{cases}
\end{equation}
Here $m_e c$ is the electron's rest energy. 
The inverse Compton effect produces the second large \emph{bump} in the SEDs of many sources.
Combining the shape of the synchrotron photon spectra with the IC photon spectra we can now draw the
SED of a typical gamma-ray source like the Crab Nebula. 
\Cref{fig:funk_sed} shows a drawing of an SED as it would be observed from a hypothetical source along with its main features and dependencies.

\begin{figure}[t]
  \centering
  \includegraphics{build/funk.pdf}
  \caption[Schematic SED model]{ A sketch of a typical SED with \ssclong emission as observed in many supernova remnants. 
  The SED was simulated using \naima~\cite{naima}. 
  The underlying electron spectrum was assumed to be distributed according to a simple power law with index $p_1=2$ before
  the electron population starts to cool and index $p_2=3$ after~\cite{kardashev}.
  The cutoff energy at which the cooling sets in is fixed at \SI{E12}{\eV}. As shown in \eqref{eq:synchrotron_spectrum} the synchrotron spectrum
  follows the electron spectrum with a modified index.
  The right side of the figure shows the inverse Compton bump. The IC emission from the CMB photons 
  and the synchrotron photons is drawn separately as dotted and dashed lines respectively. This figure was adapted from Stefan Funk's review article
  \enquote{Ground- and Space-Based Gamma-Ray Astronomy}~\cite{funk_gamma}.
  }
  \label{fig:funk_sed}
\end{figure}

\section{Log-Parabolic Energy Distributions}
\label{sec:log-par-he}
The inverse Compton emission measured by Cherenkov telescopes often shows remarkable curvature. 
Extra-galactic sources like Markarian 421 as well as SNRs within our own galaxy show a curved photon spectrum at high energies.
The approximation made in \cref{sec:ic} describing the spectrum as a power law seems to be too crude.
In the previous section I assumed an electron population that was produced by a \emph{Fermi-like}
process resulting in a power-law distribution. Each \emph{collision} with a magnetic field resulted in an energy gain of 
$\Delta \gamma = \alpha \gamma$. We can adapt the energy gain processes slightly to account for some randomness 
during the process. 
Following ideas from~\cite{acceleration_tramacere}, I consider a single charged particle in proximity to moving magnetic fields.
We can express the energy of the particle using its Lorentz factor after each collision with the magnetic field  as
\begin{align*}
  \gamma_n & = \epsilon_n \gamma_{n-1} \\
           & = \epsilon_n (\epsilon_{n-1} \gamma_{n-2})\\
           & = \epsilon_n (\epsilon_{n-1} (\epsilon_{n-2} \gamma_{n-3})) \\ 
           &\setbox0\hbox{=}\mathrel{\makebox[\wd0]{\hfil\vdots\hfil}} \\ %see https://tex.stackexchange.com/questions/7650/centering-vdots-in-a-system-of-many-equations
           & = \gamma_0 \prod_{i=1}^n \epsilon_{i},
\end{align*}
where $\epsilon_i$ is the energy gain received by the particle in collision $i$ and $\gamma_i$ is its Lorentz factor.
Suppose the particle starts with a low kinetic energy, we can set $\gamma_0 = 1$ and rearrange the equation a bit
\begin{align*}
  \gamma_n  & = \prod_{i=1}^n \epsilon_{i} \\
            & = e^{\ln(\prod_{i=1}^n \epsilon_{i})} \\
            & = e^{\sum_{i=1}^n \ln(\epsilon_{i})} \\
            & = e^{\sum_{i=1}^n X_i},
\end{align*}
where we set $X_i = \ln(\epsilon_{i})$. The $X_i$ are assumed to be identically distributed,
independent of each other and have finite variances.
This may sound like a bold claim at first. However, the mechanisms taking place in each collision are 
bound to the same physical processes and completely uncorrelated with each other.
Hence, the random variables $X_i$ fulfill the conditions for applying the central limit theorem. It follows that for large $n$ the sum  
$\chi = \sum_{i=1}^n \ln(\epsilon_{i})$ converges in distribution to a normal distribution $\chi \sim \mathcal{N}(\mu, \sigma)$ 
with $\mu = n \mu(\ln(\epsilon_i))$ and $\sigma^2 = n \sigma^2(\ln(\epsilon_i))$.
In consequence, the energy distribution of the particles $f(\gamma)$ will follow a log-normal distribution~\cite[312]{stats_degroot}
\begin{equation*}
  \ln\left(f(\gamma)\right) \sim \mathcal{N}(\mu, \sigma^2).
\end{equation*}
Transforming $\chi$ into a standard normal variable using $Z = \frac{\chi - \mu}{\sigma}$ we can write 
\begin{equation*}
  f(E) \sim e^{\mu + \sigma Z}.
\end{equation*}
 In literature the model often takes the form of an exponential function with three parameters $A$, $\alpha$ and $\beta$
\begin{equation}
  \label{eq:logpar}
  N(E) = A \left( \frac{E}{E_0} \right)^{-\alpha -\beta \log_{10}\left(\frac{E}{E_0}\right)}.
\end{equation}
As shown in~\cite[\S5]{massaro_logpar_mrk} and~\cite{lefa_kelner_shape}, the  inverse Compton emission will 
approximately follow the shape of the electron spectrum at high energies. 
Hence, a log-parabolic electron distribution will lead to a log-parabolic gamma-ray emission. 
\Cref{fig:sed_fit_he} shows a fit of \cref{eq:logpar} to observed fluxes from the Crab Nebula. 


\begin{figure}[]
  \centering
  \input{build/sed_fit_he.pgf}
  \caption[Fit to the high-energy Crab emission]{Simple least-squares fit of a log-parabolic model to observations of the Crab Nebula.
  The best fitted values are \protect{\input{build/sed_fit_he.txt}}. 
  Errors on the parameters are estimated from the diagonal of the covariance matrix resulting from the least squares fit. 
  The error band in the plot has been estimated by sampling models from a gaussian with the same covariance matrix.
  The band indicates the 5\th and 95\th percentile of all \num{10000} sampled models that are drawn from the gaussian.
  For this source the coverage of flux measurements in the transition region between synchrotron and inverse compton emission 
  is provided by the \fermi satellite. For many other sources, mostly active galactic nuclei, the transition region is shifted to lower energies.
  }
  \label{fig:sed_fit_he}
\end{figure}

